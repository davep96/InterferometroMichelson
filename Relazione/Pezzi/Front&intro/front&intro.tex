\documentclass[a4paper,11pt]{article}

\usepackage[italian]{babel}

\usepackage[latin1]{inputenc}

\usepackage[T1]{fontenc}

\usepackage{graphicx}

\usepackage{indentfirst}

\usepackage{amsmath,amssymb}

\usepackage{enumitem} 

\newcommand{\virgolette}[1]{``#1''}

\usepackage[margin=1in]{geometry} %Smaller margins

\usepackage{lmodern} %Vector PDF

\usepackage{siunitx}

\usepackage{xcolor}

\usepackage{colortbl}

\usepackage{multirow}

\usepackage{rotating}

\usepackage{booktabs}

\usepackage{longtable}

\usepackage{graphicx}
\graphicspath{ {Images/} }

\usepackage{wrapfig}

\newcommand*\chem[1]{\ensuremath{\mathrm{#1}}}

\begin{document}

\begin{titlepage}
	\centering
	{\scshape\LARGE Laboratorio di Ottica, Elettronica e \\ Fisica Moderna \par}
	\vspace{1cm}
	{\scshape\Large Relazione di Laboratorio 3\par}
	\vspace{1.5cm}
	{\huge\bfseries Interferometro di Michelson\par}
	\vspace{2cm}

	{\Large\itshape Nicolò Cavalleri, Giacomo Lini e Davide Passaro
		
	(LUN12)}

	\vspace{5cm}
	\vfill

	\begin{abstract}
	Di seguito vengono riportate la procedura sperimentale, e l'analisi dei dati raccolti relaivi a un esperimento compiuto con un Interferometro di Michelson. In particolare tramite il calcolo delle frange di interferenza dell'immagine prodotta viene determinata la lunghezza d'onda della luce emessa da un laser monocromatico. Altre grandezze fisiche rilevanti che questo apparato consente di mettere in evidenza sono l'indice di rifrazione dell'aria, la lunghezza dei pacchetti d'onda emessi da luce non moncromatica, e la differenza di lunghezza d'onda tra sorgenti differenti, nello specifico l'analisi è relativa al doppietto di sodio. Per ugnuna di queste grandezze vengono riportate procedure sperimentali e risultati comprensivi di errore.
	\end{abstract}


	\vfill
	{\large \today\par}

\end{titlepage}

\newpage

\section{Introduzione}
\end{document}
