\documentclass[a4paper,11pt]{article}

\usepackage[T1]{fontenc}

\usepackage[utf8]{inputenc}

\usepackage[italian]{babel}

\usepackage{graphicx}

\usepackage{indentfirst}

\usepackage{amsmath,amssymb}

\usepackage{enumitem} 

\newcommand{\virgolette}[1]{``#1''}

\usepackage[margin=1in]{geometry} %Smaller margins

\usepackage{lmodern} %Vector PDF

\usepackage{siunitx}

\usepackage{xcolor}

\usepackage{colortbl}

\usepackage{multirow}

\usepackage{rotating}

\usepackage{booktabs}

\usepackage{longtable}

\usepackage{graphicx}
\graphicspath{ {../../Immagini/} }

\usepackage{wrapfig}

\usepackage{siunitx} % Per unit� di misura in generale e la corretta rappresentazione dei numeri.

\usepackage{gensymb} % Per il simbolo di gradi

\begin{document}
	\label{interf} %Label di una figura schematica di un interferometro
	\label{strum} %Sezione strumentazione
	\label{calib} % subsubsection nella quale è descrtta la calibrazione dello specchio fisso
	
	\begin{equation}\label{delta}
	\Delta x = \dfrac{\lambda}{4n_a}
	\end{equation}
	
	\begin{equation}\label{lambdana}
	\lambda=\dfrac{2n_a\Delta x}{N_1}		
	\end{equation}
	\begin{equation}\label{deltavuoto}
	\Delta s = 2(n_a-1)d
	\end{equation}
	\begin{equation}\label{deltatotaria}
	2(n_a-1)d=N_2\lambda
	\end{equation}
	
	\begin{equation}\label{lambda}
	\lambda=\dfrac{2d\Delta x}{N_1 d + N_2 \Delta x}
	\end{equation}
	\begin{equation}\label{na}
	n_a=\dfrac{N_1 d}{N_1 d + N_2 \Delta x}
	\end{equation}
	
	Per una corretta visualizzazione dei numeri, con le unità di misura e gli ordini di grandezza usare il pacchetto siunitx con i comandi:\\
	
		\SI{5}{\centi\meter}
	\\
		\SI{633e-9}{\meter}
	\\
		\num{6.022e23}
	\\
	Inoltre usiamo la convenzione internazionale per cui i decimali si separano con un punto per piacere.
	

	
\end{document}
